\chapter{致\texorpdfstring{\quad}{}谢}
%把下面文字替换
白驹过隙,时光荏苒,转眼已逝三年。在整个研究生阶段以及在完成本论文的过程中,
我想我要对许多人表达我的感激之情。他们的关心、支持和帮助使我能够顺利完成这项研究工作。

首先,我要衷心感谢我的爸妈。他们始终在背后默默支持着我,给予我无尽的鼓励和理解。他们的爱和关心是我前进的动力,没有他们的支持,我无法坚持走到今天。 

我还要感谢我的导师顾正晖老师和俞祝良老师。他们给了我较为宽松的科研环境,让我能够潜心专研,在自己热爱的领域充分地施展才能、追求卓越。也感谢两位老师在论文写作过程中给予了我悉心的指导和宝贵的建议。
他们的专业知识、严谨的态度和耐心的解答让我受益匪浅。他们的学者风范以及淳淳教诲将一直激励着我在今后的道路上不断前行。 

此外,我要诚挚地感谢我的同门师兄弟们。和他们在学术上的讨论为我提供了许多宝贵的启示和帮助。我们相互鼓励、相互学习,在困难时期给予各自支持和鼓舞。没有他们的帮助和共同成长,我想我的道路一定要比想象中的还要艰难许多。

最后,我要感谢这些年勇敢走过来的那个我。感谢我对多模态、对通用人工智能那近乎着迷的热爱和追求。在长路漫漫,孤身独往的征途上,是这份热爱让我能够勇敢面对困难,持之以恒地自我驱动、推进研究。“虽感肉体之苦,未觉精神之劳”,
在这个过程中,我不断学习、成长,并且不断挑战自己的极限。我相信,这种热爱、坚持和努力将为我的未来铺就更广阔的道路。

再次对所有给予我帮助和支持的人表示由衷的感谢。你们的帮助和支持使我能够完成这篇论文,并且走向人生的新阶段。
~\\
\begin{center}
	很喜欢高达独角兽中的一句歌词:
	~\\
	“All my wish and powers, they would take me to the gate.” 
	~\\
	就以它作为我整个研究生阶段、学生时代、乃至青春年华的休止符吧,さよなら!
	~\\
\end{center}

%把上面文字替换

\begin{minipage}[t]{0.945\textwidth}%
	\begin{flushright}
		韦政松\\
%		\today\\	% 自动时间
		2023年6月3日清晨 \\	%固定时间
		于华南理工大学
		\par\end{flushright}
\end{minipage}

