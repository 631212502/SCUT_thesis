\chapter{攻读博士/硕士学位期间取得的研究成果} %博士/硕士记得选其一
\pubfont % 论文撰写规范里,这章是5号宋体,\pubfont 设置字号为5号了。但其实很多论文用小四号也OK。
一、已发表(包括已接受待发表)的论文,以及已投稿、或已成文打算投稿、或拟成文投稿的论文情况\underline{\textbf{(只填写与学位论文内容相关的部分):}}
\begin{table}
	\centering{}%
	\pubfont 
	\begin{longtable}{|>{\centering}m{0.5cm}|m{1.8cm}|>{\centering}m{2.8cm}|>{\centering}m{2.5cm}|>{\centering}m{2.2cm}|>{\centering}m{2.cm}|>{\centering}m{1cm}|}
		\hline 
		\textbf{序号} & \textbf{作者(全体作者,按顺序排列)} & \textbf{题 目} 						   & \textbf{发表或投稿刊物名称、级别} & \textbf{发表的卷期、年月、页码} & \textbf{与学位论文哪一部分(章、节)相关} &\textbf{被索引收录情况}\tabularnewline
		\hline 
		1	 & 	                    						&  	                                                    							 &                                                                    &                 &                 & \tabularnewline
		\hline 
		% 2	 & 	蒙超恒、裴海龙、程子欢						&  	Dynamic Control Allocation for A Twin Ducted Fan UAV							 & 2020 International Conference on Guidance, Navigation and Control  & 已录用,2020年8月 & 2.3、4.3和5.2节 &EI \tabularnewline
		% \hline 
	\end{longtable}
\end{table}

注:在“发表的卷期、年月、页码”栏:

1.如果论文已发表,请填写发表的卷期、年月、页码;

2.如果论文已被接受,填写将要发表的卷期、年月;

3.以上都不是,请据实填写“已投稿”,“拟投稿”。

不够请另加页。

二、与学位内容相关的其它成果(包括专利、著作、获奖项目等)

1. 韦政松,顾正晖,邓晓燕.一种医学视觉问答的实现方法、装置及存储介质[P].发明专利:202310304810.5, 第三、四章, 2023.03.24/已受理       
        
2. 韦政松,邓晓燕,黄海真,陈洲楠.一种针对复杂抽象化事物的类人化识别交互方法[P].发明专利:ZL 201910474678.6, 第五章, 2021.03.30/已授权
%注:这部分一言难尽,我努力了很久都没有把这个表做好。感觉学校给的这个表的模板非常反人类。看国外大学的博士论文,那种像参考文献著录信息那样一行一行的,比较美观。而这个框框很难放文字进去。

\normalsize % \normalsize可以将下文调回和正文一样的字号,这个随个人喜好。注释掉的话,致谢就就跟随《攻读博士/硕士学位期间取得的研究成果》的字号。