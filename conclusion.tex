\chapter{总结与展望}

\section*{总结}
医学视觉问答是近几年新兴起的一项研究,相比于传统的针对自然图像开展的基于常识的视觉问答(VQA),具备专业化知识的视觉问答模型可以真正切实地解决
用户的疑难,让VQA系统真正具备实用性和落地价值。但由于目前医学图像问答样本的稀缺以及高昂的人力标注成本,Med-VQA模型在问答性能、泛化能力以及实用性上
都还有着巨大的发展潜力。限制Med-VQA模型性能的不仅是数据的匮乏,还因为医学图像相比自然图像具有更大的噪声,关键信息占比也更小并且多为光谱单一的灰度图,
可提取的特征十分有限。因此在不断丰富样本量的同时,也需要关注如何提高模型提取特征,并且融合特征的效果和效率,即小样本下进行特征提取和特征融合的能力;
且由于问答对样本的缺乏,如何让模型在有限的文本里与图像一起建立更多、更有效的医学视觉关联,从而获得更好的开放式问答能力也十分重要。

同时,由于医学问答是一个极具风险性的信息交换场景,错误的信息相比自然图像问答更容易造成严重的后果,所以此类系统必须具备一套完善的风险评估和预测的能力,
也就是Med-VQA模型不应该仅仅关心问答预测的准确率,更需要关注模型对其预测结果有多少把握,即不确定性量化的能力。目前QA领域大部分深度网络的权值都是点估计的形式,
基于点估计的预测可能会提供虚假的高置信度的错误回答,且无法对预测结果中的不确定性进行有效度量。针对以上这些问题,本文基于增强视觉特征理论和贝叶斯不确定性估计原理,
提出了一种多编码器混合自注意力网络MEMSA对医学图像-文本多模态特征进行高效的特征提取和关联性建模,同时依据贝叶斯神经网络设计了一种贝叶斯分类器,用来估计
模型进行分类预测时的不确定性并探讨其与模型性能之间的关系。

最后,为了提高模型在真实场景下的可用性以及体验实际的医学视觉问答效果,本文还设计了可搭载不同医学视觉问答模型用于不同场景进行交互
的模态自适应系统,并融合云服务等手段,实时在线不受时间和空间约束地为使用者提供医学视觉问答服务。论文的主要工作总结如下:
\begin{enumerate}[topsep = 0 pt, itemsep= 0 pt, parsep=0pt, partopsep=0pt, leftmargin=44pt, itemindent=0pt, labelsep=6pt, label=(\arabic*)]
    \item 提出一种多编码器混合自注意网络(MEMSA)用于医学图像-文本特征提取和关系建模,在经典的特征提取网络设计的基础上,本文针对医学图像以及文本数据的特点和痛点,针对性地设计了
    用于卷积去噪的自编码器,适用于小样本学习的模型不可知元学习编码器以及具备图像-文本语义联系和良好跨模态能力的对比学习预训练模型。再对这些编码器进行集成,分别用于
    解决目前医学图像噪声大、可用信息少、小样本下多模态大模型学习不充分以及Med-VQA模型在面对开放式问答场景时缺乏良好的跨模态图像-文本关系特征抽取和建模能力等问题。本文还通过对比实验
    与目前主流的医学视觉特征提取模型以及特征融合机制进行了比较以及实例分析。通过编码器-注意力模型单一变量对照实验,实验表明MEMSA可以取得比其他方法更出色的特征提取性能以及跨模态特征融合能力,
    通过详细分析各子类问答的准确率,说明了性能提升的来源。同时通过问答样例对比凸显了MEMSA模型在语义关系建模上的能力。
    \item 基于经典贝叶斯神经网络(BNN)提出了一种用于估计医学视觉问答模型不确定性的方法以及网络模型。在经典Med-VQA分类模型中将分类器的权值由传统的点估计替换为概率分布的形式。同时阐述了局部贝叶斯进行不确定性
    估计的原理和应用性。在深度网络中,网络模型的不确定性会通过权值分布形式向后传递,局部的贝叶斯不确定性估计具有合理性以及更好的实用性。在预测时,通过变分推断和蒙特卡洛采样方法从权值分布中进行多次采样获得
    多个子网络,对它们进行集成,不仅可以获得更可靠的预测,而且可以获得预测的不确定性,也就是模型对自己预测的把握程度。通过采样频率和不确定性估计实验,阐释了增大的采样频率可以增强模型不确定性预测能力的原理机制。
    通过拒绝分类实验,发现在医学视觉问答模型中,容易错分类的往往也是具有高不确定性的样本,从而使得拒绝对高不确定性样本进行预测这一方式可以提升模型的整体的性能表现,同时,这一方式以及机制也极大地保障了医学视觉问答
    模型的可靠性和安全性。
    \item 针对视觉问答具有复杂的输入模态这一特点以及不同数据集训练的模型适配的场景不同,并且它们之间往往存在难迁移、不通用等问题设计了模态自适应交互系统。
    在传统的视觉问答路线中增加了模型控制模块和反馈回路,这一机制使得模型可以通过收集用户反馈,通过设计合适的自学习或分类算法选择合适的模型与用户进行交互,更好、更优质地解决用户的问题。
    在医学视觉问答场景下,图像输入往往各种类医学影像。为了实现良好的实时交互和学习效果,选用图像分类的方式确定交互模型。为此用不同数据集的图像数据重构了一个用于分类的新数据集,
    通过对比各种经典机器学习方法后发现,随机森林方法在这一类问题的表现上较为突出,具有最好的效果。同时为了方便用户体验和收集样本,改善模型效果,本文还将模型以及系统部署成云端服务,
    可以随时随地给用户提供医学视觉问答咨询服务。提高了医学视觉问答研究的实用性,易用性,可以让人人都拥有一个在线“医生”提供医疗诊断服务。
\end{enumerate}

\section*{展望}
目前基于多编码器、跨模态自注意力以及不确定性估计的医学视觉问答研究处于刚萌芽的阶段,相关方法甚少。本文提出基于多编码器混合自注意网络的医学视觉问答以及其不确定性研究,并通过云服务搭建模态自适应交互系统
提高了医学视觉问答模型及系统的回答交互效果并提高了其可靠性和易用性,但仍然存在一些问题,需要进一步的研究和探索:

第一,本文提出的基于多编码器混合自注意力网络MEMSA。实际上,MEMSA多编码特征之间是直接混合拼接的方式,相比于混合拼接,采用直接融合或者系数可学习的加权融合方式或许更能凸显图像-文本之间的多模态关联。因此,多编码器融合以及其配对的注意力研究
是一个值得研究的问题。

第二,本文提出的局部贝叶斯不确定估计只针对决策层,度量的信息有限,并且没有开展信号扰动,对抗攻击等系统鲁棒性实验。贝叶斯神经网络由于采用蒙特卡洛采样方法,模型预测时容易采集到
扰动信号从而获得错误的分类,抗扰动,对抗攻击性能较差。因此,通过BNN开展医学视觉问答不确定性研究,做抗扰动、抗攻击等实验是一个值得探索的方向。

第三,本文提出的在线模态自适应医学视觉问答模型以及系统仍然缺乏医学模型和优质的样本。若要在实际使用中获得良好的效果,需要源源不断的相关数据进行训练。同时模型控制器还研究最了各种机器学习算法用于
收集样本和用户反馈,迭代以提升模型的效果。