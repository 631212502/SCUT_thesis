\chapter{总结与展望}

\section*{总结}
% 医学视觉问答是一项新兴研究,相比于传统的针对自然图像开展的基于常识的视觉问答(VQA),具备专业化知识的视觉问答模型可以真正切实地解决
% 用户的疑难,让VQA系统真正具备实用性和落地价值。但由于目前医学图像问答样本的稀缺以及高昂的人力标注成本,Med-VQA模型在问答性能、泛化能力以及实用性上
% 都还有着巨大的发展潜力。限制Med-VQA模型性能的不仅是数据的匮乏,还因为医学图像相比自然图像具有更大的噪声,关键信息占比也更小并且多为单通道灰度图,
% 可提取的特征十分有限。因此在不断丰富样本量的同时,也需要关注如何提高模型提取特征,并且融合特征的效果和效率,即小样本下进行特征提取和特征融合的能力;
% 且由于问答对样本的缺乏,如何让模型在有限的文本里与图像一起建立更多、更有效的医学视觉关联,从而获得更好的开放式问答能力也十分重要。

% 同时,由于医学问答是一个极具风险性的信息交换场景,错误的信息相比自然图像问答更容易造成严重的后果,所以此类系统必须具备一套完善的风险评估和预测的能力,
% 也就是Med-VQA模型不应该仅仅关心问答预测的准确率,更需要关注模型对其预测结果有多少把握,即不确定性量化的能力。目前QA领域大部分深度网络的权值都是点估计的形式,
% 基于点估计的预测可能会提供虚假的高置信度的错误回答,且无法对预测结果中的不确定性进行有效度量。针对以上这些问题,本文基于增强视觉特征理论和贝叶斯不确定性估计原理,
% 提出了一种多编码器混合自注意力网络MEMSA对医学图像-文本多模态特征进行高效的特征提取和关联性建模,同时依据贝叶斯神经网络设计了一种贝叶斯分类器,用来估计
% 模型进行分类预测时的不确定性并探讨其与模型性能之间的关系。

% 最后,为了提高模型在真实场景下的可用性以及体验实际的医学视觉问答效果,本文还设计了可搭载不同医学视觉问答模型用于不同场景进行交互
% 的模态自适应系统,并融合云服务等手段,实时在线不受时间和空间约束地为使用者提供医学视觉问答服务。论文的主要工作总结如下:
医学视觉问答是一项新兴研究,具有十足的发展潜力。但目前由于医学视觉问答样本十分稀缺,且存在噪声大和有用信息占比低等缺陷。并且传统的Med-VQA模型无法很好地提取特征并在不同模态的特征间建立起有效的关联,同时也不能提供对其预测结果
的把握程度,也就是不确定性量化的能力。这些问题综合导致了MedVQA这项技术使用成本高,可靠性低,难以用于现实生活之中。针对这些问题,本文分别提出了一种多编码器混合自注意网络MEMSA和贝叶斯分类器BMLP用于医学视觉问答研究。也分别探讨了它们在提高Med-VQA模型的准确性,可靠性和安全性的作用。
最后搭建了一套具有模态自适应能力的云服务医学视觉问答系统。论文的主要工作总结如下。

\begin{enumerate}[topsep = 0 pt, itemsep= 0 pt, parsep=0pt, partopsep=0pt, leftmargin=0pt, itemindent=44pt, labelsep=6pt, label=(\arabic*)]
    \item 提出一种多编码器混合自注意网络(MEMSA)用于医学图像特征提取和语义关系建模。本文针对现有Med-VQA数据和模型的特点和痛点,设计了用于去噪的卷积降噪自编码器,适用于小样本学习的模型不可知元学习模型以及具备图像-文本语义关联能力的对比学习预训练模型。
    对这些编码器进行集成,通过跨模态自注意力机制抽取这些编码特征的关联组成融合特征后再进行分类和预测。本文通过对照实验分别与目前主流的医学视觉特征提取模型以及特征融合机制进行了比较分析。通过编码器-注意力模型的对照实验,MEMSA在各项指标上均有提升,
    可以取得比其他方法更出色的特征提取性能以及跨模态特征融合能力,通过详细分析各子类问答的准确率差异,说明了性能提升的来源。同时通过问答样例对比凸显了MEMSA模型在语义关系上更细粒度的建模能力。
    \item 提出了一种用于估计医学视觉问答模型不确定性的贝叶斯分类器模型BMLP。BML在经典Med-VQA分类模型中将分类器的权值由传统的点估计替换为概率分布的形式。同时阐述了局部使用BNN进行不确定性估计的原理和应用性。
    在深度网络中,网络模型的不确定性会通过权值分布形式向后传递,局部的贝叶斯不确定性估计具有合理性以及更好的实用性。在预测时,通过变分推断和蒙特卡洛采样方法从权值分布中进行多次采样获得
    多个子网络,对它们进行集成,不仅可以获得更可靠的预测,而且可以获得预测的不确定性,也就是模型对自己预测的把握程度。通过采样频率和不确定性估计关系实验,阐释了增大的采样频率可以增强模型不确定性预测能力的原理机制。
    通过拒绝分类实验,发现在医学视觉问答模型中,容易错分类的往往也是具有高不确定性的样本,从而使得拒绝对高不确定性样本进行预测这一方式可以提升模型的整体的性能表现,当拒绝比大于$50\%$时,模型已经超越了点估计下的性能。
    说明模型预测的点都在样本分布上,这一机制也极大地保障了Med-VQA模型的可靠性和安全性。
    \item 针对视觉问答具有复杂的输入模态这一特点以及不同数据集训练的模型适配的场景各不相同,它们之间存在难迁移、不通用等问题设计了模态自适应交互系统。在视觉问答路线中增加了模型控制模块和反馈回路,该设计使得模型可以通过收集用户反馈,通过合适的自学习或分类算法选择合适的模型与用户进行交互。
    在医学视觉问答场景下,图像输入往往各种类医学影像。为了实现良好的实时交互和学习效果,本文以图像类型为基准确定交互模型。用不同数据集的各类医学图像数据重构了新数据集,通过对比各种经典机器学习方法后发现,随机森林方法以大于0.98的Acc和F1值表现突出,具有最好的效果。
    同时为了方便用户体验和收集样本,改善模型效果,本文还将模型以及系统部署成云端服务,可以随时随地给用户提供医学视觉问答咨询服务。
\end{enumerate}

\section*{展望}
目前基于多编码器、跨模态自注意力以及不确定性估计的医学视觉问答研究处于刚萌芽的阶段,相关方法甚少。本文提出基于多编码器混合自注意网络的医学视觉问答以及其不确定性研究,并通过云服务搭建模态自适应交互系统
提高了医学视觉问答模型的准确性、可靠性、安全性和易用性。但仍然存在一些问题,需要进一步的研究和探索:

第一,本文提出的基于多编码器混合自注意力网络MEMSA。实际上MEMSA是一个可以适用于任何模态输入的模型,但由于数据的限制,目前的Med-VQA样本还未触及MEMSA模型的极限。如何收集更多模态的数据,融合新的结构并向着医学大模型的方向发展是一个值得研究的问题

第二,本文提出的局部贝叶斯不确定估计只针对决策层,度量的信息有限。且本文没有开展信号扰动,对抗攻击等系统鲁棒性实验。不知道BNN的抗扰动和抗攻击的能力以及系统的稳定性。因此,通过BNN开展医学视觉问答不确定性研究,做抗扰动、抗攻击等实验是一个值得探索的方向。

第三,本文提出的在线模态自适应医学视觉问答模型以及系统仍然缺乏医学模型和优质的样本。若要在实际使用中获得良好的效果,需要源源不断的相关数据进行训练。同时模型控制器还研究最了各种机器学习算法用于收集样本和用户反馈,后续通过不断的迭代提升模型的效果。